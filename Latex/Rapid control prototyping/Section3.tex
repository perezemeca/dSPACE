
\section{Rapid control prototyping}

\emph{Rapid control prototyping} (\textbf{RCP}) is one of the most efficient ways to speed up the study and development of a new product. \textbf{RCP} greatly simplifies moving from the design to the implementation stage of a control system by quickly verifying how it will respond to real-world dynamics.\cite{RCP}\newline
A great benefit of \textbf{RCP} systems is that they eliminate the tedious and error-prone process of low-level programming because they have automatic code generation capability, which gives engineers the ability to focus on the design, implementation and evaluation of the control system.\par
Several companies offer software and hardware solutions that allow the design of control systems using a block diagram programming paradigm.
Among them, MATLAB/Simulink is probably the best known and most widely used simulation software. MATLAB is a high-level computer language for algorithm development, visualization and data analysis, while Simulink is an interactive tool for modeling, simulating and analyzing dynamic systems.\\
Simulink's companion product, Real-Time Workshop (\textbf{RTW}), provides automatic ANSI-C or ADA code generation from the Simulink block diagram. \textbf{RTW} is not hardware-specific, so the generated code can be deployed on a wide range of personal computers, digital signal processors, or even microcontrollers.\par
To prevent the wasting of the development team time with hardware limitations, conventional \textbf{RCP} systems must have three key elements: \begin{enumerate}
    \item A powerful floating-point processor, several times faster than the target processor.
    \item Different types of flexible I/O.
    \item A big memory.
\end{enumerate}
Control boards, such as the DS1104, are appropriate for motion control and are fully programmable from the Simulink environment. These large-scale \textbf{RCP} systems are very powerful and suitable for applications where functionality takes priority over price, such as in research.\par
In the educational process, for example, the least efficient, cost-effective and portable \textbf{RCP} solutions are used. \textbf{RCP} systems for educational purposes should also be as simple to use as possible. If so, students can focus on designing and verifying control systems rather than learning how to operate a \textbf{RCP} system. Such \textbf{RCP} systems are difficult to find on the market, so institutions sometimes choose to develop custom internal solutions.\\